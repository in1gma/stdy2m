\documentclass[12pt]{article}
\usepackage{amssymb,latexsym,cite}

\title{Volume Extreme Ultraviolet Holographic Imaging}

\begin{document}
\maketitle
 Holographic imaging in the soft X-ray (SXR) and extreme ultraviolet (EUV) have been demonstrated in several experiments realized using EUV/SXR lasers and synchrotron sources. These include the first realization of soft X-ray laser holography at Lawrence Livermore National Laboratory using a large laser facility, and the holographic recording of biological samples and sub-micron structures using soft X-ray radiation from synchrotrons, among other experiments.  A key idea in these experiments is to use coherent short wavelength illumination to achieve a spatial resolution beyond the reach of visible light.  

Using synchrotron radiation Gabor and Fourier holograms have been demonstrated with spatial resolution below 100 nm at SXR wavelength.  Compact EUV sources based on high harmonic generation (HHG) were also used to demonstrate table-top in-line EUV holography with a resolution of 7.9 m and 0.8 m.  Time resolved holographic imaging, that exploits the short pulsewidth of the HHG sources, was also implemented to study the ultrafast dynamics of surface deformation with a lateral resolution of the order of 100 nm.  The recent development of compact coherent EUV laser sources has opened new opportunities for the implementation of novel imaging schemes with nanometer-scale resolution that fit on a table-top. In this paper, we present a proof of principle experiment in which we demonstrate that three dimensional imaging in a volume may be obtained from a single high numerical  aperture (NA) hologram obtained with a table top EUV laser. Gabor holograms were numerically reconstructed over a range of image planes by sweeping the propagation distance. This numerical sectioning technique for holography is verified to produce a robust three dimension image of a test object. 

\end{document}